\documentclass{beamer}
\usefonttheme[onlymath]{serif}
\newcommand\bmale{\fontsize{6}{7.2}\selectfont}
\newcommand\male{\fontsize{8}{7.2}\selectfont}
\newcommand\normalne{\fontsize{10}{7.2}\selectfont}
\newcommand\duze{\fontsize{12}{7.2}\selectfont}
\setbeamertemplate{caption}{\raggedright\insertcaption\par}

\mode<presentation>
{
  \usetheme{CambridgeUS}
  \usecolortheme{beaver}
}

\usepackage{ulem}
\usepackage{tabto}
\usepackage[]{algorithm2e}
\usepackage{bm}
\usepackage{lmodern}
\usepackage[T1]{fontenc}
\usepackage[polish]{babel}
\usepackage[utf8]{inputenc}
\DeclareMathOperator*{\argmin}{argmin}
\DeclareMathOperator*{\argmax}{argmax}
\selectlanguage{polish}

\title[Efektywność Inwestycji i Systemów Transportowych] % (optional, use only with long paper titles)
{Efektywność Inwestycji i Systemów Transportowych - wykłady}

\subtitle
{Wprowadzenie, organizacja}

\author[dr inż. Rafał Kucharski] % (optional, use only with lots of authors)
{dr inż. Rafał~Kucharski\inst{1}}

\institute[] % (optional, but mostly needed)
{
  \inst{1}%
  Katedra Systemów Transportowych\\
  Politechnika Krakowska
 }


\date[KST, L-2, WIL, PK] % (optional, should be abbreviation of conference name)
{Kraków, 2019}

\pgfdeclareimage[height=1cm]{university-logo}{ZSK}
 \logo{\pgfuseimage{university-logo}}

\AtBeginSubsection[]
%{
%  \begin{frame}<beamer>{Outline}
%    \tableofcontents[currentsection,currentsubsection]
%  \end{frame}
%}



\begin{document}

\begin{frame}
  \titlepage
\end{frame}

\begin{frame}{Informacje}
\begin{itemize}
\item dr inż. Rafał Kucharski, rkucharski(at)pk.edu.pl;
\item www.kst.pk.edu.pl (konsultacje, kontakt);
\item 7 wykładów
\item egzamin ustny 
\item waga wykładu $35\%$ waga ćwiczeń $65\%$
\end{itemize}
%\tableofcontents
%%  % You might wish to add the option [pausesections]
\end{frame}


\begin{frame}{Zakres}{Wykłady:}
\begin{enumerate}
\item Wstęp
\item Podaż i popyt.
\item Model czterostadiowy i jego wykorzystanie w prognozowaniu (3 wykłady)
\item Efektywność ekonomiczna
\item Wybór wariantu optymalnego - metody wielokryterialne.
\end{enumerate}
\end{frame}


\begin{frame}{Przybliżenie zagadanienia}
\begin{block}{Problem}
Wybór odpowiedniego (spośród wielu) rozwiązania danego problemu
\end{block}
\end{frame}


\begin{frame}{Przykład}
\begin{enumerate}
\item Droga rowerowa.
\item Przestrzeń piesza.
\item Uspokojenie ruchu.
\item Linia tramwajowa.
\item Tunel drogowy.
\item Obwodnica.
\item Parking.
\end{enumerate}
\end{frame}

\end{document}
