\documentclass[11pt,a4paper]{article}
\usepackage[utf8]{inputenc}
\usepackage{amsmath}
\usepackage{amsfonts}
\usepackage{amssymb}
\usepackage{graphicx}
\usepackage{paralist}
\usepackage{lmodern}
\usepackage[left=1.5cm,right=1.5cm,top=1.5cm,bottom=1.5cm]{geometry}
\usepackage[T1]{fontenc}
\usepackage[polish]{babel}
\usepackage[utf8]{inputenc}
\DeclareMathOperator*{\argmin}{argmin}
\DeclareMathOperator*{\argmax}{argmax}
\selectlanguage{polish}
\usepackage{tikz}
\usetikzlibrary{shapes,arrows,positioning}
\tikzstyle{block} = [rectangle, draw, fill=blue!20, rounded corners, align = center, minimum height=2.5cm, minimum width=4cm]
\tikzstyle{line} = [draw, thin, ->, >=stealth]
\tikzstyle{cloud} = [draw, ellipse,fill=red!20, node distance=3cm,    minimum height=2em]
\usepackage{xcolor}
\usepackage{tabto}
\usepackage{soul}
\usepackage{float}
\usepackage{enumitem}
\date{\vspace{-14ex}}
%\date{}  % Toggle commenting to test

\begin{document}
\title{\Large \textbf{Ćwiczenie projektowe} \\Analizy Mikrosymulacyjne Funkcjonowania Skrzyżowań}
\maketitle
\noindent\rule{16.9cm}{1pt}
%\vspace{-3mm}
%\\~\\
%\NumTabs{9}
%\tab\tab Imię i nazwisko: \tab \dots\dots\dots\dots\dots\dots\dots\dots\dots\dots\dots\dots\dots\dots\dots\dots\dots \\
%\tab\tab Numer albumu: \tab \dots\dots\dots\dots\dots\dots \tab Data: \tab \dots\dots\dots\dots\dots\dots\\
%\noindent\rule{16.9cm}{1pt}


\paragraph*{Problem:}
~\\
Dla zadanej \textbf{sieci drogowej} oraz popytu na przejazd przez nią, określ \textbf{wskaźniki funkcjonowania sieci} oraz ich wrażliwość na zmianę sytuacji drogowej (zmiany popytu, oraz zmiany sieci).
Wykorzystaj narzędzie do symulacji mikroskopowej ruchu drogowego (np. Vissim, Aimsun).
Wyniki analiz zapisz w formie tabel, wykresów i syntetycznego komentarza.

\paragraph*{Sieć:}
~\\
Sieć ćwiczeniowa to sekwencja odcinków łączących źródło (lewy kraniec) z celem (prawy kraniec).\\
W narożnikach sieci znajdują się odpowiednio: rejon źródłowy $o$ o określonej generacji ruchu $q$, rejon docelowy $d$ będący celem podróży dla wyjeżdżających z rejonu źródłowego.\\

\begin{table}[H]
\begin{center}
\caption{Struktura sieci}
\begin{tabular}{|c|c|c|c|c|c|}
\hline 
lp. & nazwa & długość \textit{[km]} & liczba pasów & prędkość swobodna \textit{[km/h]} \\ 
\hline 
1 & konektor źródłowy & 0.1 & - &  - \\ 
\hline 
2 & strefa akumulacji  & 0.6 & 4 &  70  \\ 
\hline 
3 & odcinek początkowy & 1 & 2  & 70  \\ 
\hline 
4 & pierwsze zwężenie & 0.1 & 2 & 70  \\ 
\hline 
5 & odcinek środkowy  & 1 & 2 & 70   \\ 
\hline 
6 & drugie zwężenie & 0.1 & 2 & 70   \\ 
\hline 
7 & odcinek końcowy & 1 & 2  & 70  \\ 
\hline 
8 & konektor docelowy & 0.1 & -  & -  \\ 
\hline 
\end{tabular} 
\end{center}
\end{table}

Sieć skonstruowana jest tak, by umożliwić analizy funkcjonowania:

\begin{compactenum}
 \item drogi jednopasowej,
 \item drogi wielopasowej, 
 \item drogi wielopasowej ze zwężeniem, 
 \item ciągu drogowego ze skrzyżowaniem bez pierwszeństwa przejazdu (podporządkowana),
 \item ciągu drogowego ze skrzyżowaniem sygnalizowanym,
 \item ciągu drogowego z sekwencją dwóch skrzyżowań sygnalizowanych (z synchronizacją, lub bez),
 \item użytkowników o różnych modelach zachowania.
 \end{compactenum}
 
 \paragraph*{Wskaźniki funkcjonowania sieci:}
~\\
Sieć będziemy oceniać obserwując wybrane wskaźniki funkcjonowania sieci, w szczególności interesują nas:
\begin{table}[H]
\centering
\label{my-label}
\begin{tabular}{c|c|c}
\textbf{lp} & \textbf{Zmienna}                   & \textbf{jednostka} \\ \hline
1  & Delay Time                & sec/km    \\
2  & Density                   & veh/km    \\
3  & Flow                      & veh/h     \\
4  & Harmonic Speed            & km/h      \\
5  & Input Count               & veh       \\
6  & Input Flow                & veh/h     \\
7  & Mean Queue                & veh       \\
8  & Mean Virtual Queue        & veh       \\
9  & Number of Stops           & \#/veh/km \\
10 & Speed                     & km/h      \\
11 & Stop Time                 & sec/km    \\
12 & Total Travel Time         & h         \\
13 & Total Travelled Distance  & km        \\
14 & Travel Time               & sec/km    \\
15 & Vehicles Inside           & veh       \\
16 & Vehicles Outside          & veh       \\
17 & Vehicles Waiting to Enter & veh    
\end{tabular}
\end{table}
 
\paragraph*{Zmienne:}
~\\
Będziemy badać jak zmieniają się powyższe wskaźniki funkcjonowania sieci w funkcji:
\begin{compactenum}
\item parametrów odcinków (przepustowość, liczba pasów, prędkość dopuszczalna)
\item parametrów skrzyżowań (czas przejazdu, sygnalizacja, organizacja ruchu)
\item popytu (liczba pojazdów, struktura rodzajowa)
\item sterowania (udział sygnału zielonego, algorytm sterowania, długość cyklu, offset)
\item modelu zachowania kierowców (czasy reakcji, losowość zachowania, itp.)
 \end{compactenum}
 

\paragraph*{Przedział zmienności:}
~\\
Dla każdej zmiennej modelu określamy jego przedział zmienności i próbkujemy ją. Proponowana metoda: pierwsza próbka dla wartości minimalnej $x_{min}$, osiem dla losowych wartości pomiędzy minimum i maksimum $rand(x_{min},x_{max})$, jedna dla wartości maksymalnej $x_{max}$.

\paragraph*{Wyniki:}
~//
Wyniki prezentujemy w formie wykresu gdzie na osi $x$ jest zmienna której wpływ analizujemy a na osi $y$ wybrany wskaźnik którego zmianę mierzymy. Np. wpływ liczby pojazdów na prędkość ruchu $k=f(v_0)$.

\section*{Zadania:}

\subsubsection{1. Zmiana warunków ruchu wraz ze wzrostem popytu.}
Zbadamy wpływ liczby pojazdów jakie chcą przejechać przez sieć (popyt) na warunki ruchu. Rozpoczynając od niewielkiej wartości 600 poj./h (10 pojazdów/minutę) stopniowo zwiększamy popyt aż do osiągnięcia maksymalnej wartości liczby pojazdów faktycznie przejeżdżających przez sieć (\textit{Flow}). Zbadaj jak w miarę wzrostu liczby pojazdów zmieniają się wartości:

\begin{compactenum}
\item Flow
\item Speed
\item Travel time
\item Vehicles Waiting to Enter
\end{compactenum}
Wyniki zapisz w formie wykresu.

\subsubsection*{2. Zmienność wyników kolejnych symulacji i szacowanie średniej procesu stochastycznego.}
Zbadamy jak zmieniają się wyniki kolejnych symulacji. Są one pojedynczymi realizacjami procesu losowego, więc wyniki różnią się. Zbadajmy dla umiarkowanie obciążonej sieci (600 poj./h) jak zmieniają się wyniki prędkości (\textit{Speed}) i czasu przejazdu (\textit{Travel time}) w kolejnych symulacjach. \par 
Wartości uzyskane w kolejnych symulacjach $x_n$ nanieś na wykres punktowy (oś $x$ to kolejne symulacje, oś $y$ to wartość prędkości lub czasu). Średnią z $n$ symulacji ($\bar{t_n} = \sum_{i=0}^n x_i /n $) w formie wykresu liniowego. Na osobnych wykresach zaznacz odchylenie standardowe z kolejnych symulacji ($\sigma_n$), oraz zbieżność średniej ($ \vert \bar{t_n} - \bar{t_n-1} \vert / \bar{t_n}$).

\subsubsection*{3. Wpływ udziału pojazdów ciężkich na warunki ruchu}
Zwiększmy liczbę pasów na wszystkich odcinkach do dwóch. W pierwszej części określmy przepustowość takiego układu analogicznie do ćwiczenia 1. 
Będziemy symulować warunki ruchu dla popytu równego 80\% tej wartości, ale z różnym udziałem pojazdów ciężkich. Stworzymy dodatkową macierz, której udział w całkowitym potoku będzie rosnąć w kolejnych symulacjach. Rozpoczniemy od wartości 1\% a skończymy na 50\%. Określmy jak wpłynie to na wybrane wskaźniki.

\subsubsection*{4. Zwężenie drogi jednopasowej.}
Dla symulowanego uprzednio układu dwupasowego środkowy odcinek zawęź do jednego pasa. Symuluj przepływ pojazdów odpowiadający przepustowości układu jednopasowego (wyniki z ćwiczenia 1). Opisz czym różnią się wyniki dla układu jednopasowego i dwupasowego ze zwężeniem.


\subsubsection*{5. Sygnalizacja na drodze dwupasowej - przepustowość.}\label{s}
Na pierwszym zwężeniu w układzie dwupasowym bez zwężenia zakoduj sygnalizację świetlną stałoczasową o cyklu równym 120s. Zbadaj dla jakiej wartości udziału sygnału zielonego $g$ przepustowość tego układu dwupasowego sygnalizowanego będzie równała się przepustowości układu jednopasowego niesygnalizowanego.

\subsubsection*{6. Sygnalizacja na drodze dwupasowej - wpływ udziału sygnału zielonego.}
Symuluj przejazd 1500 pojazdów przez układ dwupasowy sygnalizowany. Sprawdź jak zmieniają się warunki ruchu przy zmianie udziału sygnału zielonego $g$.

\subsubsection*{7. Sygnalizacja na drodze dwupasowej - wpływ długości cyklu na warunki.}
Symuluj przejazd 1500 pojazdów przez układ dwupasowy sygnalizowany o udziale sygnału zielonego $g$ zgodnym z obliczonym powyżej. Sprawdź jak zmieniają się warunki ruchu przy zmianie długości cyklu $C$ ($g$ pozostaje bez zmian).

\subsubsection*{8. Optymalna koordynacja sygnalizacji na drodze dwupasowej.}
Symuluj przejazd 1500 pojazdów przez układ dwupasowy z sekwencją dwóch skrzyżowań. Na pierwszego skrzyżowania użyj $C$ i $g$ z ćw. \ref{s}, określ offset $o$, długosć cyklu $C$ i udział zielonego $g$ na drugim skrzyżowaniu dla któRego warunki ruchu (mierzone wybranym wskaźnikiem są najlepsze).

\subsubsection*{9. Koordynacja sygnalizacji na drodze dwupasowej - wpływ offestu na warunki ruchu.}
Symuluj przejazd 1500 pojazdów przez układ dwupasowy z sekwencją dwóch skrzyżowań o parametrach $c$ i $G$ określonych w poprzednim ćwiczeniu. Określ wpływ zmiany offsetu na drugiej sygnalizacji na warunki ruchu. 


\subsubsection{Inne:} 






Wykonaj wybrane zadania:
\begin{compactenum}
\item Wpływ liczby pojazdów na czas przejazdu.
\item Wpływ prędkości w ruchu swobodnym na liczbę pojazdów jaka dojechała do celu.
\item Wpływ liczby pojazdów na prędkość.
\item Wpływ podporządkowania ruchu na stratę czasu.
\item Wpływ udziału pojazdów ciężkich na prędkość przejazdu.
\item Wpływ zwężenia drogi na przepustowość.
\item Wpływ udziału sygnału zielonego na liczbę zatrzymań.
\item Wpływ synchronizacji.
\item Wpływ parametrów modelu zachowania kierowcy (profil przyśpieszania i hamownia, czas reakcji, uprzejmość, itp.) na przepustowość.
\item itp.
\end{compactenum}



\begin{center}
 \begin{tikzpicture}

% Draw the log-normal distribution curve
% Draw the x-axis
\draw[->,black] (0,0) -- (13,0) node[anchor = south west] {$q$ };
% Draw the y-axis
\draw[->,black] (0,0) -- (0,3) node[anchor = south east] {$v_0$};
\end{tikzpicture}
\end{center}

\noindent\rule{16.9cm}{1pt}
\end{document}

